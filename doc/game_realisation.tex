\chapter{Réalisation du Programme}
\section{Structure des données}
\subsection{Structure grille}

La grille fournie en entrée est un fichier texte contenant les dimensions de la grille ainsi qu'une matrice symbolisant la grille tel que:
\begin{itemize}
	\item Un \# signifie une case vide.
    \item Un entier de signifie une case colorée primaire.
    \item Un + signifie un pont.
\end{itemize}

\begin{verbbox}
4 4
#1#3
#3#2
1#+#
2#44
\end{verbbox}
\begin{figure}[ht]
	\centering
	\theverbbox
	\caption{Exemple de grille $4\times 4$}
	\label{fig:ex_grille}
\end{figure}

Pour stocker les informations de ce fichier, nous avons décidé d'utiliser une liste de $n$ sous-liste où $n$ est le nombre de cases de la grille.

Ces sous-listes contiennent un quadruplet et une liste d'adjacence.\\

Le quadruplet contient:
\begin{itemize}
	\item Un entier correspondant au numéro de la case.
    \item Un entier correspondant à la couleur de la case.
    \item Un entier correspondant à la couleur du pont ou "nil" si la case n'est pas un pont.
    \item Un booléen indiquant si la case est primaire.
\end{itemize}

La liste d'adjacence contient la liste des cases adjacentes à la case dont il est question.\\

Cette transformation des données est effectuée par une fonction de parcours. Elle parcourt la structure et construit de manière récursive notre structure de données. 

\subsection{Structure de remplissage}

Nous avons implémenté une deuxième structure afin de faciliter l'affichage et la validation de la grille dans l'interface graphique.

Cette structure est une liste de $m$ sous-listes où $m$ est le nombre de couleurs de la grille.

Chaque sous-liste contient:
\begin{itemize}
	\item Un entier correspondant à la couleur du chemin décrit.
    \item Des couples d'entier correspondant aux coordonnées d'une case.
\end{itemize}

Les sous-listes représentent les chemins de chacune des couleurs.

Le premier couples de coordonnées correspont à une case primaire. Les couples suivants correspondent au chemin tracé de la couleur.\\

\section{Mise en oeuvre de l'algorithme de résolution}

\subsection{Tests et complexité}

La résolution fonctionne pour des grilles de toute les tailles. Cependant certaines configurations de grilles mettent plus de temps que les autres:\\
\begin{itemize}
	\item Dans le cas de grilles dépassant $8\times 8$
    \item Dans le cas ou le nombre de couleur est faible par rapport aux dimensions de la grille. Par exemple une grille $8\times 8$ avec $4$ couleurs.
\end{itemize}

La résolution est de complexité $\Theta(C(n+m))$ où $n$ est le nombre de cases, $m$ le nombre d'arrêtes (liaison entre deux cases adjacentes) et $C$ un facteur qui dépend du nombre de recherches effectué par l'algorithme.

Le nombre de recherches effectué par l'algorithme est dans le pire des cas celui d'un algorithme de force brute.
